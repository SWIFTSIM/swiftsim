\subsection{Cosmological factors for properties entering the artificial viscosity}
\label{ssec:artificialvisc}

There are multiple properties that enter into the more complex artificial
viscosity schemes, such as those by \citet{Morris1997} (henceforth M\&M) and
\citet{Cullen2010} (henceforth C\&D).

\subsubsection{M\&M basic scheme}
\label{sssec:mandm}

This relies on the velocity divergence as a shock indicator, i.e. the property
$\nabla \cdot \mathbf{v}$. The interpretation of this is the velocity divergence of
the fluid overall, i.e. the physical velocity divergence. Starting with
\begin{equation}
\mathbf{v}_p = a \dot{\mathbf{r}}' + \dot{a} \mathbf{r}', \nonumber
\end{equation}
with the divergence,
\begin{equation}
\nabla \cdot \mathbf{v}_p =
    \nabla \cdot \left(a \dot{\mathbf{r}}'\right) +
    \nabla \cdot \left(\dot{a} \mathbf{r}'\right). \nonumber
\end{equation}
The quantity on the left is the one that we want to enter the source term for the
artificial viscosity. Transforming to the co-moving derivative on the right hand side
to enable it to be calculated in the code,
\begin{equation}
\nabla \cdot \mathbf{v}_p = 
    \nabla' \cdot \dot{\mathbf{r}}' + n_d H(a),
\label{eqn:divvwithcomovingcoordinates}
\end{equation}
with $n_d$ the number of spatial dimensions, and the final transformation
being the one to internal code velocity units,
\begin{equation}
\nabla \cdot \mathbf{v}_p = 
    \frac{1}{a^2} \nabla' \cdot \mathbf{v} + n_d H(a).
\label{eqn:divvcodeunits}
\end{equation}
We note that there is no similar hubble flow term in the expression for
$\nabla \times \mathbf{v}_p$.