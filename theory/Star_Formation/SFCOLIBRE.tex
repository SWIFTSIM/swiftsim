\section{Star Formation in COLIBRE}

In this section we will shortly explain how the star formation subgrid 
recipe work in the COLIBRE model. The mass converted to stars in COLIBRE
is given by:
\begin{align}
 \dot{m}_\star = f_\star \frac{m_\text{gas}}{\tau_\text{ff}}.
\end{align}
In which $f_\star$ is the star formation efficiency, $m_\text{gas}$ is 
the gas mass and $\tau_\text{ff}$ is the free fall time, given by:
\begin{align}
 \tau_\text{ff} = \sqrt{\frac{3 \pi}{32\rho G}}.
\end{align}
In which $G$ is the gravitational constant and $\rho$ is the physical 
density of the gas particle. No correction is made for the amount of gas 
in the hot phase. Similar to the EAGLE star formation, the probability of 
forming a star is given by:
\begin{align}
 \text{Prob.} = \text{min} \left( \frac{\dot{m}_\star \Delta t}{m_g}, 1 \right) 
 = \text{min} \left( \frac{f_\star \Delta t}{\tau_\text{ff}}, 1 \right).
\end{align}
To prevent spurious star formation at high redshifts we set a over density
threshold similar to EAGLE to prevent star formation in underdense regions 
in the early Universe. This means that the density needs to satisfy:
\begin{align}
 \delta > \delta_\text{thres.} = 57.7.
\end{align}
Besides this to form stars we also have a temperature threshold. Depending if 
we run with an EOS or not we have a difference. In the case that we do not 
have an EOS, we have a regular temperature threshold given by a constant 
temperature:
\begin{align}
 T < T_\text{thres.}.
\end{align}
If we run with an EOS this criteria will not work, therefore in this case 
we use a constant dex offset from the EOS similar to \citet{dallavecchia2012}.
This means that the criteria is given by:
\begin{align}
 \log_{10} T < \log_{10} T_\text{eos} + \Delta_\text{dex}.
\end{align}
In the case of high densities we want a high density threshold which directly
converts the gas particles to stars:
\begin{align}
 \rho > \rho_\text{high den. thresh.}.
\end{align}
Summarizing this means that hte COLIBRE star formation has the following free
parameters: $f_\star$, $\delta_\text{thres.}$, $T_\text{thres.}$, 
$\Delta_\text{dex}$ and $\rho_\text{high den. thresh.}$.
