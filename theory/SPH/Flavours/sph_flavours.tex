For vector quantities, we define $\vec{a}_{ij} \equiv (\vec{a}_i -
\vec{a}_j)$. Barred quantities ($\bar a_{ij}$) correspond to the
average over two particles of a quantity: $\bar a_{ij} \equiv
\frac{1}{2}(a_i + a_j)$. To simplify notations, we also define the
vector quantity $\Wij \equiv \frac{1}{2}\left(W(\vec{x}_{ij}, h_i) +
\nabla_x W(\vec{x}_{ij},h_j)\right)$.


%##############################################################################

\subsection{\MinimalSPH}
\label{sec:sph:minimal}

This is the simplest fully-conservative version of SPH using the
internal energy $u$ as a thermal variable that can be written
down. Its implementation in \swift should be understood as a text-book
example and template for more advanced implementations. A full
derivation and motivation for the equations can be found in the review
of \cite{Price2012}. Our implementation follows their equations (27),
(43), (44), (45), (101), (103) and (104) with $\beta=3$ and
$\alpha_u=0$. We summarize the equations here.

\subsubsection{Density and other fluid properties (\nth{1} neighbour loop)}

For a set of particles $i$ with positions $\vec{x}_i$ with velocities
$\vec{v}_i$, masses $m_i$, sthermal energy per unit mass $u_i$ and
smoothing length $h_i$, we compute the density for each particle:

\begin{equation}
  \rho_i \equiv \rho(\vec{x}_i) = \sum_j m_j W(\vec{x}_{ij}, h_i),
  \label{eq:sph:minimal:rho}
\end{equation}
and the derivative of its density with respect to $h$:

\begin{equation}
    \label{eq:sph:minimal:rho_dh}
  \rho_{\partial h_i} \equiv \dd{\rho}{h}(\vec{x}_i) = \sum_j m_j
\dd{W}{h}(\vec{x}_{ij}
  , h_i).
\end{equation}
This corresponds to $x_i = \tilde{x}_i = m_i$, and $y_i =\tilde{y}_i = \rho_i$
in the \citet{hopkins2013} formalism.  The gradient terms (``h-terms'') can
then be computed from the density and its derivative:

\begin{equation}
  f_i \equiv \left(1 + \frac{h_i}{3\rho_i}\rho_{\partial h_i}
  \right)^{-1}.
  \label{eq:sph:minimal:f_i}
\end{equation}
Using the pre-defined equation of state, the pressure $P_i$ and the sound
speed $c_i$ at the location of particle $i$ can now be computed from
$\rho_i$ and $u_i$:

\begin{align}
  P_i &= P_{\rm eos}(\rho_i, u_i),   \label{eq:sph:minimal:P}\\
  c_i &= c_{\rm eos}(\rho_i, u_i).   \label{eq:sph:minimal:c}
\end{align}

\subsubsection{Hydrodynamical accelerations (\nth{2} neighbour loop)}

We can then proceed with the second loop over neighbours. The signal velocity
$v_{{\rm sig},ij}$ between two particles is given by

\begin{align}
  \mu_{ij} &=
  \begin{cases}
  \frac{\vec{v}_{ij} \cdot \vec{x}_{ij}}{|\vec{x}_{ij}|}  & \rm{if}~
  \vec{v}_{ij} \cdot \vec{x}_{ij} < 0,\\
    0 &\rm{otherwise}, \\
  \end{cases}\nonumber\\
  v_{{\rm sig},ij} &= c_i + c_j - 3\mu_{ij}.   \label{eq:sph:minimal:v_sig}
\end{align}
We also use these two quantities for the calculation of the viscosity term:

\begin{equation}
\nu_{ij} = -\frac{1}{2}\frac{\alpha \mu_{ij} v_{{\rm
      sig},ij}}{\bar\rho_{ij}}
  \label{eq:sph:minimal:nu_ij}
\end{equation}
The fluid accelerations are then given by

\begin{align}
  \frac{d\vec{v}_i}{dt} = -\sum_j m_j &\left[\frac{f_iP_i}{\rho_i^2}
  \nabla_x W(\vec{x}_{ij}, h_i)   \nonumber
  +\frac{f_jP_j}{\rho_j^2}\nabla_x W(\vec{x}_{ij},h_j)\right. \\
  &+ \left. \bigg.\nu_{ij} \Wij \right], \label{eq:sph:minimal:dv_dt}
\end{align}
and the change in internal energy,

\begin{align}
  \frac{du_i}{dt} = \sum_j m_j &\left[\frac{f_iP_i}{\rho_i^2}  \vec{v}_{ij}
    \cdot \nabla_x W(\vec{x}_{ij}, h_i) \right. \label{eq:sph:minimal:du_dt}\\
    &+\left. \frac{1}{2}\nu_{ij}\vec{v}_{ij}\cdot\Big. \Wij\right], \nonumber
\end{align}
where in both cases the first line corresponds to the standard SPH
term and the second line to the viscous accelerations.

We also compute an estimator of the change in smoothing length to be
used in the prediction step. This is an estimate of the local
divergence of the velocity field compatible with the accelerations
computed above:

\begin{equation}
  \frac{dh_i}{dt} = -\frac{1}{3}h_i \sum_j \frac{m_j}{\rho_j}
  \vec{v}_{ij}\cdot \nabla_x W(\vec{x}_{ij}, h_i).
  \label{eq:sph:minimal:dh_dt}
\end{equation}
and update the signal velocity of the particles:

\begin{equation}
  v_{{\rm sig},i} = \max_j \left( v_{{\rm sig},ij} \right).
  \label{eq:sph:minimal:v_sig_update}
\end{equation}
All the quantities required for time integration have now been obtained.

\subsubsection{Time integration}

For each particle, we compute a time-step given by the CFL condition:

\begin{equation}
  \Delta t = 2 C_{\rm CFL} \frac{H_i}{v_{{\rm sig},i}},
    \label{eq:sph:minimal:dt}
\end{equation}
where $C_{\rm CFL}$ is a free dimensionless parameter and $H_i = \gamma h_i$ is
the kernel support size. Particles can then be ``kicked'' forward in time:
\begin{align}
  \vec{v}_i &\rightarrow \vec{v}_i + \frac{d\vec{v}_i}{dt} \Delta t 
\label{eq:sph:minimal:kick_v}\\
  u_i &\rightarrow u_i + \frac{du_i}{dt} \Delta t
\label{eq:sph:minimal:kick_u}\\
  P_i &\rightarrow P_{\rm eos}\left(\rho_i, u_i\right)
\label{eq:sph:minimal:kick_P}, \\
  c_i &\rightarrow c_{\rm eos}\left(\rho_i, u_i\right)
\label{eq:sph:minimal:kick_c},
\end{align}
where we used the pre-defined equation of state to compute the new
value of the pressure and sound-speed.

\subsubsection{Particle properties prediction}

Inactive particles need to have their quantities predicted forward in
time in the ``drift'' operation. We update them as follows:

\begin{align}
  \vec{x}_i &\rightarrow \vec{x}_i + \vec{v}_i \Delta t 
\label{eq:sph:minimal:drift_x} \\
  h_i &\rightarrow h_i \exp\left(\frac{1}{h_i} \frac{dh_i}{dt}
  \Delta t\right), \label{eq:sph:minimal:drift_h}\\
  \rho_i &\rightarrow \rho_i \exp\left(-\frac{3}{h_i} \frac{dh_i}{dt}
  \Delta t\right), \label{eq:sph:minimal:drift_rho} \\
  P_i &\rightarrow P_{\rm eos}\left(\rho_i, u_i + \frac{du_i}{dt} \Delta
t\right), \label{eq:sph:minimal:drift_P}\\
  c_i &\rightarrow c_{\rm eos}\left(\rho_i, u_i + \frac{du_i}{dt}
  \Delta t\right) \label{eq:sph:minimal:drift_c},
\end{align}
where, as above, the last two updated quantities are obtained using
the pre-defined equation of state. Note that the thermal energy $u_i$
itself is \emph{not} updated.

%##############################################################################

\subsection{Gadget-2 SPH}
\label{sec:sph:gadget2}

This flavour of SPH is the one implemented in the \gadget-2 code
\citep{Springel2005}. The basic equations were derived by
\cite{Springel2002} and also includes a \cite{Balsara1995} switch for
the suppression of viscosity. The implementation here follows closely the
presentation of \cite{Springel2005}. Specifically, we use their equations (5),
(7), (8), (9), (10), (13), (14) and (17). We summarise them here for
completeness.

\subsubsection{Density and other fluid properties (\nth{1} neighbour loop)}

For a set of particles $i$ with positions $\vec{x}_i$ with velocities
$\vec{v}_i$, masses $m_i$, entropic function per unit mass $A_i$ and
smoothing length $h_i$, we compute the density, derivative of the density with
respect to $h$ and the ``h-terms'' in a similar way to the minimal-SPH case
(Eq. \ref{eq:sph:minimal:rho}, \ref{eq:sph:minimal:rho_dh} and
\ref{eq:sph:minimal:f_i}). From these the pressure and sound-speed can
be computed using the pre-defined equation of state:

\begin{align}
  P_i &= P_{\rm eos}(\rho_i, A_i),   \label{eq:sph:gadget2:P}\\
  c_i &= c_{\rm eos}(\rho_i, A_i).   \label{eq:sph:gadget2:c}
\end{align}
We additionally compute the divergence and curl of the velocity field using
standard SPH expressions:

\begin{align}
  \nabla\cdot\vec{v}_i &\equiv\nabla\cdot \vec{v}(\vec{x}_i) = \frac{1}{\rho_i}
\sum_j m_j
  \vec{v}_{ij}\cdot\nabla_x W(\vec{x}_{ij}, h_i)
\label{eq:sph:gadget2:div_v},\\ 
    \nabla\times\vec{v}_i &\equiv \nabla\times \vec{v}(\vec{x}_i) =
\frac{1}{\rho_i} \sum_j m_j
  \vec{v}_{ij}\times\nabla_x W(\vec{x}_{ij}, h_i) \label{eq:sph:gadget2:rot_v}.
\end{align}
These are used to construct the \cite{Balsara1995} switch $B_i$:

\begin{equation}
  B_i = \frac{|\nabla\cdot\vec{v}_i|}{|\nabla\cdot\vec{v}_i| +
    |\nabla\times\vec{v}_i| + 10^{-4}c_i / h_i}, \label{eq:sph:gadget2:balsara}
\end{equation}
where the last term in the denominator is added to prevent numerical
instabilities.

\subsubsection{Hydrodynamical accelerations (\nth{2} neighbour loop)}

The accelerations are computed in a similar fashion to the minimal-SPH
case with the exception of the viscosity term which is now modified to
include the switch. Instead of Eq. \ref{eq:sph:minimal:nu_ij}, we get:

\begin{equation}
\nu_{ij} = -\frac{1}{2}\frac{\alpha \bar B_{ij} \mu_{ij} v_{{\rm
sig},ij}}{\bar\rho_{ij}},
  \label{eq:sph:gadget2:nu_ij}  
\end{equation}
whilst equations \ref{eq:sph:minimal:v_sig},
\ref{eq:sph:minimal:dv_dt}, \ref{eq:sph:minimal:dh_dt} and
\ref{eq:sph:minimal:v_sig_update} remain unchanged. The only other
change is the equation of motion for the thermodynamic variable which
now has to be describing the evolution of the entropic function and
not the evolution of the thermal energy. Instead of eq.
\ref{eq:sph:minimal:du_dt}, we have

\begin{equation}
\frac{dA_i}{dt} = \frac{1}{2} A_{\rm eos}\left(\rho_i, \sum_j
m_j \nu_{ij}\vec{v}_{ij}\cdot \Wij\right),
\end{equation}
where we made use of the pre-defined equation of state relating
density and internal energy to entropy.

\subsubsection{Time integration}

The time-step condition is identical to the \MinimalSPH case
(Eq. \ref{eq:sph:minimal:dt}). The same applies to the integration
forward in time (Eq. \ref{eq:sph:minimal:kick_v} to
\ref{eq:sph:minimal:kick_c}) with the exception of the change in
internal energy (Eq. \ref{eq:sph:minimal:kick_u}) which gets replaced
by an integration for the the entropy:


\begin{align}
  \vec{v}_i &\rightarrow \vec{v}_i + \frac{d\vec{v}_i}{dt} \Delta t 
\label{eq:sph:gadget2:kick_v}\\
  A_i &\rightarrow A_i + \frac{dA_i}{dt} \Delta t
\label{eq:sph:gadget2:kick_A}\\
  P_i &\rightarrow P_{\rm eos}\left(\rho_i, A_i\right)
\label{eq:sph:gadget2:kick_P}, \\
  c_i &\rightarrow c_{\rm eos}\left(\rho_i, A_i\right)
\label{eq:sph:gadget2:kick_c},
\end{align}
where, once again, we made use of the equation of state relating
thermodynamical quantities.

\subsubsection{Particle properties prediction}

The prediction step is also identical to the \MinimalSPH case with the
entropic function replacing the thermal energy.

\begin{align}
  \vec{x}_i &\rightarrow \vec{x}_i + \vec{v}_i \Delta t 
\label{eq:sph:gadget2:drift_x} \\
  h_i &\rightarrow h_i \exp\left(\frac{1}{h_i} \frac{dh_i}{dt}
  \Delta t\right), \label{eq:sph:gadget2:drift_h}\\
  \rho_i &\rightarrow \rho_i \exp\left(-\frac{3}{h_i} \frac{dh_i}{dt}
  \Delta t\right), \label{eq:sph:gadget2:drift_rho} \\
  P_i &\rightarrow P_{\rm eos}\left(\rho_i, A_i + \frac{dA_i}{dt} \Delta
t\right), \label{eq:sph:gadget2:drift_P}\\
  c_i &\rightarrow c_{\rm eos}\left(\rho_i, A_i + \frac{dA_i}{dt}
  \Delta t\right) \label{eq:sph:gadget2:drift_c},
\end{align}
where, as above, the last two updated quantities are obtained using
the pre-defined equation of state. Note that the entropic function $A_i$
itself is \emph{not} updated.

\subsection{Weighted-Pressure SPH Validity}

A new class of Lagrangian SPH methods were introduced to the astrophysical
community by \citet{Hopkins2013} and \citet{Saitoh2013}. Two of these methods,
Pressure-Entropy (used in the original ANARCHY implementation in EAGLE) and
Pressure-Energy, are implemented for use in \swift{}. Before considering the
use of these methods, though, it is important to pause for a moment and
consider where it is valid to use them in a cosmological context. These methods
(as implemented in \swift{}) are only valid for cases that use an \emph{ideal
gas equation of state}, i.e. one in which
\begin{equation}
  P = (\gamma - 1) u \rho \propto \rho.
  \nonumber
\end{equation}
Implementations that differ from this, such as the original ANARCHY scheme in
EAGLE, may have some problems with energy conservation \cite[see][]{Hosono2013}
and other properties as at their core they assume that pressure is proportional
to the local energy density, i.e. $P \propto \rho u$. This is most easily shown
in Pressure-Energy SPH where the weighted pressure $\bar{P}$ is written as
\begin{equation}
  \bar{P} = \sum_j \frac{P_j}{\rho_j} W_{ij} = \sum_j m_j (\gamma - 1) u_j
W_{ij},
\end{equation}
and the right-hand side is what is actually calculated using the scheme. It is 
clear that this does not give a valid weighted pressure for any scheme using a
non-ideal equation of state. Fortunately, there is a general prescription for
including non-ideal equations of state in the P-X formalisms, but this is yet
to be implemented in \swift{} and requires an extra density loop. Attempting to
use these weighted schemes with a non-ideal equation of state will lead to an
incorrect calculation of both the pressure and the equation fo motion. How
incorrect this estimate is, however, remains to be seen.

%##############################################################################

\subsection{Pressure-Entropy SPH}
\label{sec:sph:pe}

This flavour of SPH follows the implementation described in section
2.2.3 of \cite{Hopkins2013}. We start with their equations (17), (19),
(21) and (22) but modify them for efficiency and generality
reasons. We also use the same \cite{Balsara1995} viscosity switch as
in the \GadgetSPH scheme (Sec. \ref{sec:sph:gadget2}).

\subsubsection{Density and other fluid properties (\nth{1} neighbour loop)}

For a set of particles $i$ with positions $\vec{x}_i$ with velocities
$\vec{v}_i$, masses $m_i$, entropic function per unit mass $A_i$ and
smoothing length $h_i$, we compute the density, derivative of the
density with respect to $h$, divergence and curl of velocity field in
a similar fashion to the \GadgetSPH scheme. From the basic particle
properties we construct an additional temporary quantity

\begin{equation}
  \tilde{A_i} \equiv A_i^{1/\gamma},
    \label{eq:sph:pe:A_tilde}
\end{equation}
which enters many equations. This allows us to construct the
entropy-weighted density $\bar\rho_i$:

\begin{equation}
  \bar\rho_i = \frac{1}{\tilde{A_i}} \sum_j m_j \tilde{A_j} W(\vec{x}_{ij},
h_i),
  \label{eq:sph:pe:rho_bar}
\end{equation}
which can then be used to construct an entropy-weighted sound-speed
and pressure based on our assumed equation of state:

\begin{align}
  \bar c_i &= c_{\rm eos}(\bar\rho_i, A_i), \label{eq:sph:pe:c_bar}\\
  \bar P_i &= P_{\rm eos}(\bar\rho_i, A_i), \label{eq:sph:pe:P_bar}
\end{align}
and estimate the derivative of this later quantity with respect to the
smoothing length using:

\begin{equation}
\bar P_{\partial h_i} \equiv \dd{\bar{P}}{h}(\vec{x}_i) = \sum_j m_j
\tilde{A_j} \dd{W}{h}(\vec{x}_{ij}), \label{eq:sph:pe:P_dh}
\end{equation}
This corresponds to $x_i = m_i \tilde{A}_i$, $\tilde{x}_i = m_i$, $y_i =
\bar{P}_i$, and $\tilde{y}_i = \rho_i$ in the \citet{hopkins2013} formalism.
The gradient terms (``h-terms'') are then obtained by combining $\bar
P_{\partial h_i}$ and $\rho_{\partial h_i}$ (eq. \ref{eq:sph:minimal:rho_dh}):

\begin{align}
    f_{ij} = & ~ 1 - \tilde{A}_j^{-1} f_i \nonumber \\
    f_i \equiv &  \left(\frac{h_i}{3\rho_i}\bar P_{\partial
    h_i}\right)\left(1 + \frac{h_i}{3\rho_i}\rho_{\partial
    h_i}\right)^{-1}. 
\end{align}

\subsubsection{Hydrodynamical accelerations (\nth{2} neighbour loop)}

The accelerations are given by the following term:

\begin{align}
  \frac{d\vec{v}_i}{dt} = -\sum_j m_j &\left[\frac{\bar P_i}{\bar\rho_i^2}
\left(\frac{\tilde A_j}{\tilde A_i} - \frac{f_i}{\tilde A_i}\right)\nabla_x
W(\vec{x}_{ij}, h_i) \right.  \nonumber \\
  &+\frac{P_j}{\rho_j^2} \left(\frac{\tilde A_i}{\tilde A_j} -
\frac{f_j}{\tilde A_j}\right)\nabla_x W(\vec{x}_{ij},h_j) \\
  &+ \left. \bigg.\nu_{ij} \Wij \right], \label{eq:sph:pe:dv_dt}
\end{align}
where the viscosity term $\nu_{ij}$ has been computed as in
the \GadgetSPH case (Eq. \ref{eq:sph:gadget2:balsara}
and \ref{eq:sph:gadget2:nu_ij}). For completeness, the equation of
motion for the entropy is

\begin{equation}
\frac{dA_i}{dt} = \frac{1}{2} A_{\rm eos}\left(\rho_i, \sum_j
m_j \nu_{ij}\vec{v}_{ij}\cdot \Wij\right).
\end{equation}

\subsubsection{Time integration}

The time-step condition is identical to the \MinimalSPH case
(Eq. \ref{eq:sph:minimal:dt}). The same applies to the integration
forward in time (Eq. \ref{eq:sph:minimal:kick_v} to
\ref{eq:sph:minimal:kick_c}) with the exception of the change in
internal energy (Eq. \ref{eq:sph:minimal:kick_u}) which gets replaced
by an integration for the the entropy:

\begin{align}
  \vec{v}_i &\rightarrow \vec{v}_i + \frac{d\vec{v}_i}{dt} \Delta t 
\label{eq:sph:pe:kick_v}\\
  A_i &\rightarrow A_i + \frac{dA_i}{dt} \Delta t \label{eq:sph:pe:kick_A}\\
  P_i &\rightarrow P_{\rm eos}\left(\rho_i, A_i\right)
\label{eq:sph:pe:kick_P}, \\
  c_i &\rightarrow c_{\rm eos}\left(\rho_i,
  A_i\right) \label{eq:sph:pe:kick_c}, \\
  \tilde A_i &= A_i^{1/\gamma}
\end{align}
where, once again, we made use of the equation of state relating
thermodynamical quantities.


\subsubsection{Particle properties prediction}

The prediction step is also identical to the \MinimalSPH case with the
entropic function replacing the thermal energy.

\begin{align}
  \vec{x}_i &\rightarrow \vec{x}_i + \vec{v}_i \Delta t 
\label{eq:sph:pe:drift_x} \\
  h_i &\rightarrow h_i \exp\left(\frac{1}{h_i} \frac{dh_i}{dt}
  \Delta t\right), \label{eq:sph:pe:drift_h}\\
  \rho_i &\rightarrow \rho_i \exp\left(-\frac{3}{h_i} \frac{dh_i}{dt}
  \Delta t\right), \label{eq:sph:pe:drift_rho} \\
  \tilde A_i &\rightarrow \left(A_i + \frac{dA_i}{dt}
  \Delta t\right)^{1/\gamma} \label{eq:sph:pe:drift_A_tilde}, \\
  P_i &\rightarrow P_{\rm eos}\left(\rho_i, A_i + \frac{dA_i}{dt} \Delta
t\right), \label{eq:sph:pe:drift_P}\\
  c_i &\rightarrow c_{\rm eos}\left(\rho_i, A_i + \frac{dA_i}{dt}
  \Delta t\right) \label{eq:sph:pe:drift_c}, 
\end{align}
where, as above, the last two updated quantities are obtained using
the pre-defined equation of state. Note that the entropic function $A_i$
itself is \emph{not} updated.

\subsection{Pressure-Energy SPH}
\label{sec:sph:pu}

Section 2.2.2 of \cite{Hopkins2013} describes the equations for Pressure-Energy
(P-U) SPH; they are reproduced here with some more details.

P-U SPH depends on the calculation of a smoothed pressure, and follows the
evolution of the internal energy, as opposed to the entropy.

For P-U, the following choice of parameters in the formalism of \S
\ref{sec:derivation} provides convenient properties:
\begin{align}
  x_i =&~ (\gamma - 1) m_i u_i, \\
  \tilde{x}_i =&~ 1,
  \label{eq:sph:pu:xichoice}
\end{align}
leading to the following requirements to ensure correct volume elements:
\begin{align}
  y_i =& \sum_{j} (\gamma - 1) m_j u_j W_{ij} = \bar{P}_i,\\
  \tilde{y}_i =& \sum_{j} W_{ij} = \bar{n}_i,
  \label{eq:sph:pu:yichoice}
\end{align}
with the resulting variables representing a smoothed pressure and particle
number density. This choice of variables leads to the following equation of
motion:
\begin{align}
  \frac{\mathrm{d} \mathbf{v}_i}{\mathrm{d} t} = -\sum_j (\gamma - 1)^2 m_j u_j
u_i
	 &\left[\frac{f_{ij}}{\bar{P}_i} \nabla_i W_{ij}(h_i) ~+ \right. \nonumber \\
	       &\frac{f_{ji}}{\bar{P}_j} \nabla_i W_{ji}(h_j) ~+ \nonumber \\
	       & \left.\nu_{ij}\bar{\nabla_i W_{ij}}\right]~.
  \label{eq:sph:pu:eom}
\end{align}
which includes the Monaghan artificial viscosity term and Balsara switch in
the final term.

The $h$-terms are given as
\begin{align}
  f_{ij} = 1 - \left[\frac{h_i}{n_d (\gamma - 1) \bar{n}_i \left\{m_j
u_j\right\}}
		   \frac{\partial \bar{P}_i}{\partial h_i} \right]
		   \left( 1 + \frac{h_i}{n_d \bar{n}_i}
		   \frac{\partial \bar{n}_i}{\partial h_i} \right)^{-1}
  \label{eq:sph:pu:fij}
\end{align}
with $n_d$ the number of dimensions. In practice, the majority of $f_{ij}$ is
precomputed in {\tt hydro\_prepare\_force} as only the curly-bracketed term
depends on the $j$ particle. This cuts out on the majority of operations,
including expensive divisions.

In a similar fashion to \MinimalSPH, the internal energy must also be
evolved. Following \cite{Hopkins2013}, this is calculated as
\begin{align}
  \frac{\mathrm{d}u_i}{\mathrm{d}t} = \sum_j (\gamma - 1)^2 m_j u_j u_i
	\frac{f_{ij}}{\bar{P}_i}(\mathbf{v}_i - \mathbf{v}_j) \cdot
	\nabla_i W_{ij}(h_i)~.
  \label{eq:sph:pu:dudt}
\end{align}
The sound-speed in P-U requires some consideration. To see what the `correct'
sound-speed
is, it is worth looking at the equation of motion (Equation
\ref{eq:sph:pu:eom}) in
contrast with that of the EoM for Density-Energy SPH (Equation
\ref{eq:sph:minimal:dv_dt})
to see what terms are applicable. For Density-Energy SPH, we see that
\begin{align}
  \frac{\mathrm{d}\mathbf{v}_i}{\mathrm{d} t} \sim \frac{c_{s, i}}{\rho_i}
\nabla_i W_{ij},
  \nonumber
\end{align}
and for Pressure-Energy SPH
\begin{align}
  \frac{\mathrm{d}\mathbf{v}_i}{\mathrm{d} t} \sim (\gamma - 1)^2
  \frac{u_i u_j}{\bar{P}_i} \nabla_i W_{ij}.
  \nonumber
\end{align}
From this it is reasonable to assume that the sound-speed, i.e. the speed at
which information propagates in the system through pressure waves, is given by
the expression
\begin{align}
  c_{s, i} = (\gamma - 1) u_i \sqrt{\gamma \frac{\rho_i}{\bar{P_i}}}.
  \label{eq:sph:pu:soundspeedfromeom}
\end{align}
This expression is dimensionally consistent with a sound-speed, and includes
the gas density information (through $\rho$), traditionally used for
sound-speeds, as well as including the extra information from the smoothed
pressure $\bar{P}$. However, this scheme causes some problems, which can be
illustrated using the Sedov Blast test. Such a sound-speed leads to a
considerably \emph{higher} time-step in front of the shock wave (where the
smoothed pressure is higher, but the SPH density is relatively constant),
leading to integration problems. An alternative to this is to use the smoothed
pressure in the place of the ``real" pressure. Whilst it is well understood
that $\bar{P}$ should not be used to replace the real pressure in general, here
(in the sound-speed) it is only used as part of the time-stepping condition.
Using
\begin{align}
  c_{s, i} = \sqrt{\gamma \frac{\bar{P}_i}{\rho_i}}
  \label{eq:sph:pu:soundspeed}
\end{align}
instead of Equation \ref{eq:sph:pu:soundspeedfromeom} leads to a much improved
time-stepping condition that actually allows particles to be woken up before
being hit by a shock (see Figure \ref{fig:sph:soundspeed}).

\begin{figure}
  \centering
  \includegraphics[width=\columnwidth]{sedov_blast_soundspeed.pdf}
  \caption{The ratio of the sound-speed calculated in Equation
    \ref{eq:sph:pu:soundspeed} to the gas sound-speed,
    $c_s = \sqrt{\gamma(\gamma - 1) u}$ with $u$ the internal energy. Note how
    the sound-speed increases ahead of the shock, leading to a much smaller
    time-step for these particles ($\Delta t \propto c_s^{-1}$), waking them up
    just before they are hit by a shock. The physical reasoning behind using
    this particular metric for the sound-speed is weak, but as a time-step
    criterion it appears to be useful. The smoothed pressure calculation
    ``catches" the hot particles from the shock that is incoming and is
    increased for those in front of the shock. Thankfully, particles that are
    behind the shock seem to be relatively unaffected. The red line shows the
    shock front.}
  \label{fig:sph:soundspeed}  
\end{figure}


\subsubsection{Time integration}

Time integration follows exactly the same scheme as \MinimalSPH, as the
fundamental equations are exactly the same (just slightly different quantities
enter the equations of motion). The CFL condition is used, along with the
sound-speed that is discussed above, such that
\begin{align}
  \Delta t = 2 C_{\rm CFL} \frac{H_i}{v_{{\rm sig},i}},
  \label{eq:sph:pu:dt}
\end{align}
where $C_{\rm CFL}$ is a free dimensionless parameter and $H_i = \gamma h_i$ is
the kernel support size. There is an additional requirement placed on the
maximal change in $u$ such that
\begin{align}
  \Delta t = C_{u} \frac{u}{du/dt},
  \label{eq:sph:pu:dt_du}
\end{align}
with $C_{u}$ a free dimensionless parameter that describes the maximal change
in $u$ that is allowed.

\subsubsection{Particle properties prediction}

The prediction of particle properties follows exactly the same scheme as
\MinimalSPH, with the exception of course of the additional smoothed quantity
$\bar{P}$. This is drifted in the same way as the density; they should both
follow from the same differential equation with an additional $u$ factor on
both sides, such that
\begin{align}
  \bar{P}_i \rightarrow \bar{P}_i
  \exp\left(-\frac{3}{h_i}\frac{dh_i}{dt} \Delta t\right). 
  \label{eq:sph:pu:drift}
\end{align}

%##############################################################################

\subsection{Variable artificial viscosity}

Here we consider a modified version of the Pressure-Energy scheme described
above but one that uses a variable artificial viscosity. The prescription used
in this scheme was originally introduced by \citet{Morris1997} and is almost
identical to the above equations, but tracks an individual viscosity paramaeter
$\alpha_i$ for each particle. This viscosity is then updated each time-step to
a more appropraite value. The hope is that the artificial viscosity will be
high in regions that contain shocks, but as low as possible in regions where it
is uneccesary such as shear flows. This is already accomplished somewhat with
the inclusion of a \citet{Balsara1995} switch, but a fixed $\alpha$ still leads
to spurious transport of angular momentum and vorticity.

The equation governing the growth of the viscosity is
\begin{align}
  \frac{\mathrm{d} \alpha_i}
       {\mathrm{d} t} = 
  - (\alpha_i - \alpha_{\rm min}) \ell \frac{c_{s, i}}{h},
  \label{eq:sph:pu:alphadt}
\end{align}
with $\alpha_{\rm min}=0.1$ the minimal artificial viscosity parameter, and
$\ell=0.1$ the viscosity ``length" that governs how quickly the viscosity
decays. This equation is solved implicitly in a similar way to
$\mathrm{d}\mathbf{v}/ \mathrm{d}t$ and $\mathrm{d}u/\mathrm{d}t$ - i.e.
$\alpha_{i} (t+\Delta t_i) = \alpha_{i}(t) + \dot{\alpha}_i \Delta t_i$.

To ensure that the scheme is conservative, the viscosity coefficients must be
combined in a fully conservative way; this is performed by taking the mean
viscosity parameter of the two particles that are being interacted, such that
\begin{align}
  \alpha_{ij} = \frac{\alpha_i + \alpha_j}{2}.
\end{align}
The rest of the artificial viscosity implementation, including the
\citet{Balsara1995} switch, is the same - just with $\alpha \rightarrow
\alpha_{ij}$.

\subsection{Anarchy SPH}
Dalla Vecchia (\textit{in prep.}), also described in section 2.2.2 of
\cite{Schaller2015}.\\
\label{sec:sph:anarchy}
\tbd 
